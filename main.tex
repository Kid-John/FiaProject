\documentclass[acmsmall, 10pt]{acmart}
\bibliographystyle{ACM-Reference-Format}
\usepackage{graphicx} % Required for inserting images
\usepackage{hyperref}

\title{Progetto FIA LikingSongs}
\author{Citro Giovanni}
\date{January 2024}

\begin{document}

\maketitle

\section{Definizione del Problema}
Il sistema prevede un modello di classificazione in cui le canzoni vengono messe nei preferiti, quindi viene deciso dall'utente se una canzone piace(1) o no(0) e a quel punto, tramite i dati ottenuti dall'utente, il sistema sarà in grado di predire se una canzone piace o meno all'utente in questione.
Il dataset è stato reperibile grazie al sito Kaggle.
\\
\section{Specifica dell'ambiente}
\subsection{Performance}
La misura di performance dell’agente è la sua capacità di avvicinarsi quanto
più possibile ad una situazione ideale nella quale vengono mostrati agli utenti
esattamente il valore che ci indica se una canzone piace(1) o no(0) in base al dataset con l'aggiunta della colonna Liking.
\subsection{Envoirenment}

\subsection{Actuators}
Gli attuatori dell’agente consistono nella lista di canzoni con un parametro che ci dice quale canzone può piacere e quale no.

\subsection{Sensors}
ciao
\\
\section{Raccolta, Analisi e Scrematura del dataset}
\subsection{Scelta del dataset}
Per i dataset del modello di classificazione ho usato la strada più semplice, ovvero quella di cercare un dataset con le canzoni di spotify e i suoi attributi. Infine ho creato un dataset inserendo la colonna che mi serviva per far apprendere al modello i valori di gradimento da predire.
\subsection{Analisi e scrematura del dataset}
Guardando bene il dataset ho visto che alcune colonne erano inutili al mio intento e quindi sono andato ad eliminarle(added, top year, acous, spch, pop, dB, live, nrgy, dnce, bpm), poi ho formattato alcuni valori all'interno di alcune colonne come: Anno di Rilascio, Durata e Tipo di Artista.
\\
Il progetto completo fatto su colab è presente sul mio \href{https://github.com/Kid-John/FiaProject.git}{github}.



\end{document}
